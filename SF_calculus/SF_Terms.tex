\documentclass[12pt]{report}
\usepackage[]{inputenc}
\usepackage[T1]{fontenc}
\usepackage{fullpage}
\usepackage{coqdoc}
\usepackage{amsmath,amssymb}
\usepackage{url}
\begin{document}
%%%%%%%%%%%%%%%%%%%%%%%%%%%%%%%%%%%%%%%%%%%%%%%%%%%%%%%%%%%%%%%%%
%% This file has been automatically generated with the command
%% coqdoc SF_Terms.v --latex --no-lib-name -o SF_Terms.tex 
%%%%%%%%%%%%%%%%%%%%%%%%%%%%%%%%%%%%%%%%%%%%%%%%%%%%%%%%%%%%%%%%%
\coqlibrary{SF Terms}{}{SF\_Terms}

\begin{coqdoccode}
\coqdocemptyline
\coqdocemptyline
\end{coqdoccode}
The operators S and F are enough, but more may be added in future. \begin{coqdoccode}
\coqdocemptyline
\coqdocnoindent
\coqdockw{Inductive} \coqdocvar{operator} := \ensuremath{|} \coqdocvar{Sop} \ensuremath{|} \coqdocvar{Fop} .\coqdoceol
\coqdocemptyline
\end{coqdoccode}
The terms of SF-calculus are either variables (given as de Bruijn indices), operators or applications. 
Terms are called combinations if they do not use any variables. \begin{coqdoccode}
\coqdocemptyline
\coqdocnoindent
\coqdockw{Inductive} \coqdocvar{SF}:  \coqdockw{Set} :=\coqdoceol
\coqdocindent{1.00em}
\ensuremath{|} \coqdocvar{Ref} : \coqdocvar{nat} \ensuremath{\rightarrow} \coqdocvar{SF}        \coqdoceol
\coqdocindent{1.00em}
\ensuremath{|} \coqdocvar{Op}  : \coqdocvar{operator} \ensuremath{\rightarrow} \coqdocvar{SF}   \coqdoceol
\coqdocindent{1.00em}
\ensuremath{|} \coqdocvar{App} : \coqdocvar{SF} \ensuremath{\rightarrow} \coqdocvar{SF} \ensuremath{\rightarrow} \coqdocvar{SF}   \coqdoceol
\coqdocnoindent
.\coqdoceol
\coqdocemptyline
\end{coqdoccode}
\end{document}
